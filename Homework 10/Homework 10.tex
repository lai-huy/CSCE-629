\documentclass{article}
\usepackage{amsmath,amssymb,amsthm,latexsym,paralist,url}

\theoremstyle{definition}
\newtheorem{problem}{Problem}
\newtheorem*{solution}{Solution}
\newtheorem*{resources}{Resources}

\newcommand{\name}[1]{\noindent\textbf{Name:} {#1}}
\newcommand{\honor}{\noindent On my honor, as an Aggie, I have neither
  given nor received any unauthorized aid on any portion of the
  academic work included in this assignment. Furthermore, I have
  disclosed all resources (people, books, web sites, etc.) that have
  been used to prepare this homework. \\[1ex]
 \textbf{Signature:} \underline{\hspace*{5cm}} }
 
\newcommand{\checklist}{\noindent\textbf{Checklist:}
\begin{compactitem}[$\Box$] 
\item Did you add your name? 
\item Did you disclose all resources that you have used? \\
(This includes all people, books, websites, etc. that you have consulted)
\item Did you sign that you followed the Aggie honor code? 
\item Did you solve all problems? 
\item Did you submit the pdf file resulting from your latex file 
  of your homework?
\end{compactitem}
}

\newcommand{\problemset}[1]{\begin{center}\textbf{Problem Set #1}\end{center}}
\newcommand{\duedate}[1]{\begin{quote}\textbf{Due date:} Electronic
    submission of this homework is due on \textbf{#1} on canvas. In
    order to receive any credit, you
    need to typeset your homework in LaTeX and submit the resulting
    pdf file. Any submission that cannot be checked for plagiarism will not
    be graded.\end{quote}}

\newcommand{\N}{\mathbf{N}}
\newcommand{\R}{\mathbf{R}}
\newcommand{\Z}{\mathbf{Z}}


\begin{document}
\problemset{10}
\duedate{4/21/2023}
\name{ Huy Quang Lai}
\begin{resources} Cormen, Thomas H., Leiserson, Charles E., Rivest, Ronald L., Stein, Clifford. \textit{Introduction to Algorithms}. The MIT Press.
\end{resources}
\honor

\newpage

Read the chapter on Approximation Algorithms in our textbook and study
the slides on the Polynomial Hierarchy. 


\begin{problem} (20 points) 
Prove by induction that if \textbf{P}=\textbf{NP}, then
\textbf{PH}=\textbf{P}. [Hint: Study the slides on the Polynomial Hierarchy]
\end{problem}
\begin{solution}
Assume that $\textbf{P}=\textbf{NP}$\\
By induction we will prove that $\Sigma_i,\Pi_i=\textbf{P}$ by induction for all $i>0$.

\noindent
Base Case:\\
$i=1$\\
$\Pi_1=\textbf{coNP}$ in the class of languages $L$ for which $L^c\in\textbf{NP}$.\\
Since $\textbf{P}=\textbf{NP}$, $L^c\in\textbf{P}$\\
Therefore, $\Pi_1=\textbf{P}=\textbf{NP}$.

\noindent
By assumption, $\Sigma_1=\textbf{P}=\textbf{NP}$

\noindent
Inductive Case:\\
Assume that $\Sigma_i,\Pi_i=\textbf{P}$\\
A language $L\in\Sigma_{i+1}$ if $\exists L'\in\Pi_i\land c>0$ such that
\[x\in L \Leftrightarrow\exists y,|y|<|x|^c,(x,y)\in L'\]

\noindent
By the inductive hypothesis, $L'\in \textbf{P}$. Therefore, there exists a polynomial-time algorithm to compute it. However, $L\in \textbf{NP}$\\
Since $\textbf{P}=\textbf{NP},L\in \textbf{P}$.
Therefore, $\Sigma_{i+1}=\textbf{P}$

\noindent
A similar proof exists to show that $\Pi_{i+1}=\textbf{P}$

\noindent
A language $L\in\Pi{i+1}$ if $\exists L'\in\Sigma_i\land c>0$ such that
\[x\in L \Leftrightarrow\exists y,|y|<|x|^c,(x,y)\in L'\]

\noindent
By the inductive hypothesis, $L'\in \textbf{P}$. Therefore, there exists a polynomial-time algorithm to compute it. However, $L\in \textbf{NP}$\\
Since $\textbf{P}=\textbf{NP},L\in \textbf{P}$.
Therefore, $\Sigma_{i+1}=\textbf{P}$
\end{solution}

\clearpage
\begin{problem} (30 points) 
Let $C_n$ denote the cycle graph on $n$ vertices, where $n$ is a
positive integer satisfying $n\ge 3$.
\begin{compactenum}[(a)]
\item What is the size of a maximum independent set in $C_n$?
\item For which $n$ can the approximation algorithm 
APPROX-VERTEX-COVER (see our textbook) give an optimal result?
[Hint: You can use your insights from part (a).] 
\end{compactenum}
\end{problem}
\noindent
\begin{solution}
Part (a)\\
An independent set in a graph is a set of vertices in which no two vertices are adjacent. 
In a cycle graph $C_n$, any two adjacent vertices form an independent set of size $2$. 
Moreover, any vertex in $C_n$ is adjacent to two other vertices. 
Therefore, any independent set in $C_n$ can contain at most one vertex from each pair of adjacent vertices.

\noindent
If $n$ is odd, then $C_n$ has an odd number of vertices, say $n=2k+1$, and the maximum independent set is obtained by selecting every other vertex along the cycle. 
This gives a set of size $k+1$, because there are $k+1$ pairs of adjacent vertices in $C_n$.

\noindent
If $n$ is even, then $C_n$ has an even number of vertices, say $n=2k$, and the maximum independent set is obtained by selecting every other vertex along the cycle, starting with either the first or second vertex. 
This gives a set of size $k$, because there are $k$ pairs of adjacent vertices in $C_n$.

\noindent
Therefore, the size of a maximum independent set in $C_n$ is $\displaystyle\left\lceil \frac{n}{2} \right\rceil$, where $\lceil x \rceil$ denotes the smallest integer greater than or equal to $x$.

\noindent
Part (b)\\
First, note that any vertex cover of $C_n$ must contain at least one vertex from each pair of adjacent vertices, because otherwise the two adjacent vertices are not covered.
Therefore, $C^*$ contains at least $\displaystyle\left\lceil \frac{n}{2} \right\rceil$ vertices, which is the number of pairs of adjacent vertices in $C_n$.

\noindent
Now consider the vertices selected by the algorithm.
Let $v_1,v_2,\ldots,v_k$ be the vertices selected in order, and let $E_i$ be the set of edges incident to $v_i$ that are not yet covered after step $i$.
Note that $|E_1|=\operatorname{deg}(v_1)$, and $|E_i|\le\operatorname{deg}(v_i)$ for $i>1$, because each edge incident to $v_i$ is adjacent to a vertex that has already been covered.
Therefore, we have
\[
|C|=k\leq\sum_{i=1}^{k}|E_i|\leq\sum_{i=1}^{k}\operatorname{deg}(v_1)
\]
where the last inequality follows from $|E_i| \le \operatorname{deg}(v_i)$.

\noindent
On the other hand, since $C^*$ is a vertex cover, we have
\[
\sum_{v\in C^*}^{}\operatorname{deg}(v)\geq n
\]
because every edge is incident to at least one vertex in $C^*$.
Therefore, if $\sum_{v \in C^*} \operatorname{deg}(v) = n$, then $C^*$ is a minimum vertex cover and the algorithm will output $C^*$ as well.

\noindent
If $n$ is odd, then $C_n$ has $\lfloor n/2 \rfloor$ pairs of adjacent vertices, and the maximum degree of any vertex is $2$. Therefore, the sum of degrees in $C_n$ is $n$, and $C^*$ is a minimum vertex cover. In this case, the algorithm will output a minimum vertex cover and the result will be optimal.

\noindent
If $n$ is even, then $C_n$ has $\displaystyle \frac{n}{2}$ pairs of adjacent vertices, and the maximum degree of any vertex is $2$.
Therefore, the sum of degrees in $C_n$ is $n$, and $C^*$ is a minimum vertex cover if and only if $C^*$ contains exactly one vertex from each pair of adjacent vertices.
If $C^*$ contains more than one vertex from any pair of adjacent vertices, then we can obtain a smaller vertex cover by removing one of the vertices.
In this case, the algorithm will output a vertex cover of size $\displaystyle\left\lceil \frac{n}{2} \right\rceil$, which is at most one larger than the minimum vertex cover.
Therefore, the algorithm will give an optimal result if and only if $C^*$ contains exactly one vertex from each pair of adjacent vertices.
\end{solution}

\begin{problem} (30 points)
  (a) Give an efficient greedy algorithm that finds an optimal vertex
  cover for a tree in linear time. (b) Explain why your algorithm
  works and justify your claim about the run-time. 
\end{problem}
\begin{solution}
Part (a)
\begin{enumerate}
    \item Start at the root node of the tree.
    \item Traverse the tree using depth-first search (DFS) and maintain two sets of nodes: $S$ (the vertex cover) and $T$ (the set of nodes not yet covered).
    \item When visiting a node $v$, add it to $S$ and remove $v$ and its neighbors from $T$.
    \item Continue DFS until all nodes have been visited.
\end{enumerate}

\noindent
Part (b)\\
This algorithm works because a tree is a special case of a graph where each edge connects exactly two nodes.
Hence, for any edge in the tree, one of its endpoints must be included in the vertex cover.
Therefore, we can use a greedy approach and include each node as we traverse the tree.

\noindent
Moreover, the algorithm's runtime is linear, $O(n)$, where $n$ is the number of nodes in the tree.
This is because we visit each node once, and each node takes constant time to process (adding to $S$ and removing from $T$).
Additionally, the set operations can be performed in constant time using hash tables or other efficient data structures.
Therefore, the overall runtime of the algorithm is $O(n)$.
\end{solution}

\clearpage
\begin{problem} (20 points) From the lecture (see also the chapter on
  NP-completeness in our textbook for details), we know that the
  vertex-cover problem and the NP-complete clique problem are
  complementary in the sense that an optimal vertex cover is the
  complement of a maximum-size clique in the complement graph. Does
  this relationship imply that there is a polynomial-time
  approximation algorithm with a constant approximation ratio for the
  clique problem? Justify your answer.
\end{problem}
\begin{solution}
No, the relationship between the vertex cover problem and the clique problem does not imply the existence of a polynomial-time approximation algorithm with a constant approximation ratio for the clique problem.

\noindent
Although an optimal vertex cover is the complement of a maximum-size clique in the complement graph, this relationship only provides a way to obtain a lower bound for the clique problem from an upper bound for the vertex cover problem.
It does not give any direct insight into the approximability of the clique problem.

\noindent
The clique problem is known to be NP-complete, and there is no known polynomial-time approximation algorithm with a constant approximation ratio for it, unless $\textbf{P}=\textbf{NP}$. In fact, it is known that there is no polynomial-time approximation algorithm for the clique problem with an approximation ratio better than $\displaystyle O\left(n^{\frac{1}{2}-\epsilon}\right)$, for any $\epsilon>0$, unless $\textbf{P}=\textbf{NP}$.

\noindent
Therefore, the relationship between the vertex cover problem and the clique problem does not imply the existence of a polynomial-time approximation algorithm with a constant approximation ratio for the clique problem.
\end{solution}
\end{document}
