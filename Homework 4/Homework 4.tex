\documentclass{article}
\usepackage{amsmath,amssymb,amsthm,latexsym,paralist}

\theoremstyle{definition}
\newtheorem{problem}{Problem}
\newtheorem*{solution}{Solution}
\newtheorem*{resources}{Resources}

\newcommand{\name}[1]{\noindent\textbf{Name: #1}}
\newcommand{\honor}{\noindent On my honor, as an Aggie, I have neither
  given nor received any unauthorized aid on any portion of the
  academic work included in this assignment. Furthermore, I have
  disclosed all resources (people, books, web sites, etc.) that have
  been used to prepare this homework. The solutions given in this
  homework are my own work.\\[1ex]
 \textbf{Signature:} \underline{\hspace*{5cm}} }

 \newcommand{\checklist}{\noindent\textbf{Checklist:}
\begin{compactitem}[$\Box$] 
\item Did you add your name? 
\item Did you disclose all resources that you have used? \\
(This includes all people, books, websites, etc. that you have consulted)
\item Did you sign that you followed the Aggie honor code? 
\item Did you solve all problems? 
\item Did you submit the pdf file of your homework?
\end{compactitem}
}



\newcommand{\problemset}[1]{\begin{center}\textbf{Problem Set
      #1}\end{center}}
\newcommand{\duedate}[2]{\begin{quote}\textbf{Due dates:} Electronic
    submission of the pdf file of this homework is due on
    \textbf{#1} on canvas. \end{quote} }

\newcommand{\N}{\mathbf{N}}
\newcommand{\R}{\mathbf{R}}
\newcommand{\Z}{\mathbf{Z}}


\begin{document}

\problemset{4}
\duedate{2/17/2023 before 11.59pm}{}
\name{Huy Quang Lai}
\begin{resources} Cormen, Thomas H., Leiserson, Charles E., Rivest, Ronald L., Stein, Clifford. \textit{Introduction to Algorithms}. The MIT Press.
\end{resources}
\honor

\newpage
Make sure that you describe all solutions in your own words. Typeset
your solutions in \LaTeX. Read
chapters 30 and 15 in our textbook. 

\begin{problem} (20 points) Let $\omega$ be a primitive $n$th root of unity. 
The fast Fourier transform implements the multiplication with
  the matrix 
$$ F = (\omega^{ij})_{i,j\in [0..n-1]}.$$
Show that the inverse of the matrix $F$ is given by 
$$ F^{-1} = \frac{1}{n}  (\omega^{-jk})_{j,k\in [0..n-1]}$$
[Hint: $x^n-1= (x-1)(x^{n-1}+\cdots + x + 1),$ so any power
$\omega^\ell\neq 1$  must be a root of $x^{n-1}+\cdots + x + 1$.]  
Thus, the inverse FFT, called IFFT, is nothing but the FFT using
$\omega^{-1}$ instead of $\omega$, and multiplying the result with
$1/n$. 
\end{problem}
\begin{solution}
Since $F$ is a matrix, by definition, $F\cdot F^{-1}=I$ where $I$ is the identity matrix.\\
Representing $F$ as a summation, leads to
\[F=\sum_{j=0}^{n-1}\omega^{ij}=\begin{cases}
n & \text{if $i\mod{n}=0$} \\
0 & \text{if $i\mod{n}\neq0$}
\end{cases}\]

\noindent
This summation is valid because if $i$ is a multiple of $n$, $\omega^i=1$. Since the summation has a lower-bound of $0$ and a higher bound of $n-1$, there will be $n$ ones added together resulting in $n$.\\
$F=0$ when $i$ is not a multiple of $n$ using the hint given.\\
The summation would result in $\omega^{0i}+\omega^{1i}+\cdots+\omega^{(n-1)i}$. From the hint, when this summation is multiplied by $\omega^i-1$ the result will be $\omega^n-1$.\\
Since $\omega^{in}-1=(\omega^i-1)\left(\omega^{0i}+\omega^{1i}+\cdots+\omega^{(n-1)i}\right)$, all $\omega^i$'s are roots of this equation since any $\omega^i$ plugged into the left-hand side would result in the $n$th root raised to the $n$th power. As a result, the left-hand side is $1-1=0$.

\noindent
From this result, either $\omega^i-1$ or $\omega^{0i}+\omega^{1i}+\cdots+\omega^{(n-1)i}$ must be equal to zero.\\
Since $\omega^i\neq1$, then $\omega^{0i}+\omega^{1i}+\cdots+\omega^{(n-1)i}$ must be equal to zero.

\noindent
Next $F^{-1}$ needs a summation representation. By definition, $F\cdot F^{-1}=I$. Using the summation definition, $F\cdot F^{-1}=1$.\\
In order to get this result, the imaginary numbers can be multiplied by their complex conjugates to result in $1$.
This complex conjugate would result in the following summation,
\[F^{-1}=\sum_{j=0}^{n-1}\omega^{-jk}\]
Multiplying $F$ and $F^{-1}$, the product should be $1$.
\[F\cdot F^{-1}=\sum_{i=0}^{n-1}\omega^{(i-k)j}\]

\noindent
This summation is very similar to the representation of $F$ with $i$ substituted for $i-k$. 
As a result, $F\cdot F^{-1}$ will equal $n$ if $i-k$ is a multiple of $n$. However we need $F\cdot F^{-1}$ to equal zero in this condition, therefore the entire summation needs to be multiplied by $\displaystyle\frac{1}{n}$.

\noindent
Finally, with these results,
\[F^{-1}=\frac{1}{n}\left(\omega^{-jk}\right)_{j,k\in[0..n-1]}\]

\end{solution}

\begin{problem} (20 points) 
Describe in your own words how to do a polynomial multiplication using the FFT and
  IFFT for polynomials $A(x)$ and $B(x)$ of degree $\le n-1$. Make
  sure that you describe the length of the FFT and IFFT needed for
  this task. Be concise and precise. 
\end{problem}
\begin{solution}
Since $A(x),B(x)\in P_{n-1}$,\\
$A(x)=a_0+a_1x+a_2x^2+\cdots+a_{n-1}x^{n-1},B(x)=b_0+b_1x+b_2x^2+\cdots+b_{n-1}x^{n-1}$.
Representing the polynomials as vectors results in,
\[A=\langle a_0,a_1,a_2,\cdots a_{n-1}\rangle, B=\langle b_0,b_1,b_2,\cdots b_{n-1}\rangle\]

\noindent
Setting up Fast Fourier Transform, $A^*$ and $B^*$ need to be split into their respective odd and even degrees.\\
Let $A_{even}=\langle a_0,a_2,\cdots,a_{n-2}\rangle,A_{odd}=\langle a_1,a_3,\cdots,a_{n-1}\rangle$ and a similar definition for $B_{even}$ and $B_{odd}$.\\
With these polynomials, $A(x)=A_{even}(x^2)+xA_{odd}(x^2)$.\\
Since the even and odd polynomials only have half of the degree of the original polynomial, $x^2$ is needed to compensate.\\
By doing this division, the number of operations is reduced to $\displaystyle\frac{n}{2}$ every recursive call.\\
Converting $A$ into $A^*$, where $A^*$ is the vector of point values for the polynomial $A(x)$. A similar process is done for $B(x)$.

\noindent
Then, to multiply the two polynomials, multiply each corresponding point value together. This would result in some vector $C*$ that holds the point values of the product polynomial. After $C^*$ is calculated, a inverse conversion is needed.\\
To do this, an IFFT would need to be applied. The IFFT would have the same things as the FFT except $\omega^i$ is replaced with its complex conjugate.\\
Then, the resulting vector would need to be divided by $n$ and result in $C$, the coefficient vector representation of the product polynomial, $C(x)$.\\
Each conversion would take $O(n\log n)$ time, and each root of unity computation would be $O(1)$.
\end{solution}

\clearpage
\begin{problem} (20 points) 
How can you modify the polynomial multiplication algorithm based
  on FFT and IFFT to do multiplication of long integers in base 10?
  Make sure that you take care of carries in a proper way. Write your
  algorithm in pseudocode and give a brief explanation. 
\end{problem}
\begin{solution}
The polynomial multiplication can be modified to multiply long integers in base ten by representing each digit of the integers as coefficients of the polynomials.\\
For example, the polynomial for $123456$ would be $a_0=6,a_1=5,a_2=4,a_3=3,a_4=2,a_5=1$ This process is repeated for the second number.

\noindent
After the conversion, the FFT and IFFT from problem 2 is applied to the polynomials.\\
Once the polynomial product is calculated, multiply each coefficient by $10^i$ where i is the coefficient number. Then add all the coefficients after being multiplied together to get the product.

\begin{verbatim}
def mult(A, B) {
    // Convert A and B into 
    A_poly = to_polynomial(A)
    B_poly = to_polynomial(B)

    // Apply FFT on A and B
    A' = FFT(A_poly)
    B' = FFT(B_poly)

    // Multiplly A' and B'
    C' = pointwise_mult(A', B')

    // Apply IFFT
    C = IFFT(C')

    // Calculate product
    sum = 0
    for (int i = 0; i < C.size(); ++i)
        sum += C[i] * 10^i
    return sum
}
\end{verbatim}
\end{solution}

\clearpage
\begin{problem}[20 points]
  Solve Exercise 15.3-4 on page 389 of our textbook. As stated, in
  dynamic programming we first solve the subproblems and then choose
  which of them to use in an optimal solution to the
  problem. Professor Capulet claims that we do not always need to
  solve all the subproblems in order to find an optimal solution. She
  suggests that we can find an optimal solution to the matrix-chain
  multiplication problem by always choosing the matrix $A_k$
  at which to split the subproduct $A_iA_{i+1}\cdots A_j$ (by selecting $k$ to minimize the
  quantity $p_{i-1}p_kp_j$) before solving the subproblems. Find an instance
  of the matrix-chain multiplication problem for which this greedy
  approach yields a suboptimal solution.
\end{problem}
\begin{solution}
A matrix-chain multiplication problem that yields a sub-optimal solution is the three matrices that follows.\\
$A_0$, a $5\times6$ matrix; $A_1$, a $6\times3$ matrix; and $A_2$, a $3\times2$ matrix.\\
Using the greedy approach, the multiplication would be $(A_0A_1)A_2=120$ multiplications. However the optimal solution is $A_0(A_1A_2)=96$ multiplications.
\end{solution}

\clearpage
\begin{problem}[20 points]
  Solve Exercise 15.4-1 on page 396 of our textbook.
  Determine an LCS of $\langle 1,0,0,1,0,1,0,1\rangle$ and
  $\langle0,1,0,1,1,0,1,1,0\rangle$. Make sure that you explain your
  answer step-by-step, in detail, rather than just giving an LCS. 
\end{problem}
\begin{solution}
Let $X=\langle 1,0,0,1,0,1,0,1\rangle$\\
Let $Y=\langle0,1,0,1,1,0,1,1,0\rangle$\\
Using the following algorithm, the length of the LCS between the two sequences can be determined.
\begin{enumerate}
    \item Initialize a table T with 0's of size $(m+1)\times(n+1)$, where $m$ and $n$ are the lengths of the sequences $X$ and $Y$, respectively.
    \item Iterate through the sequences $X$ and $Y$, and for each pair of elements $(x,y)$ do the following:
    \begin{enumerate}
        \item If $x$ equals $y$, set $T[i][j]=T[i-1][j-1]+1$
        \item If $x$ does not equal $y$, set $T[i][j]=\max(T[i-1][j],T[i][j-1])$
    \end{enumerate}
    \item The length of the LCS is given by $T[m][n]$.
\end{enumerate}

\noindent
Filling out the table using this algorithm results in\\
\begin{tabular}{ll|lllllllll}
  &    & $X$  & 1 & 0 & 0 & 1 & 0 & 1 & 0 & 1 \\
  &    & -1 & 0 & 1 & 2 & 3 & 4 & 5 & 6 & 7 \\ \hline
$Y$ & -1 & 0  & 0 & 0 & 0 & 0 & 0 & 0 & 0 & 0 \\
0 & 0  & 0  & 0 & 1 & 1 & 1 & 1 & 1 & 1 & 1 \\
1 & 1  & 0  & 1 & 1 & 1 & 2 & 2 & 2 & 2 & 2 \\
0 & 2  & 0  & 1 & 2 & 2 & 2 & 3 & 3 & 3 & 3 \\
1 & 3  & 0  & 1 & 2 & 2 & 3 & 3 & 4 & 4 & 4 \\
1 & 4  & 0  & 1 & 2 & 2 & 3 & 3 & 4 & 4 & 5 \\
0 & 5  & 0  & 1 & 2 & 3 & 3 & 4 & 4 & 5 & 6 \\
1 & 6  & 0  & 1 & 2 & 3 & 4 & 4 & 5 & 5 & 6 \\
1 & 7  & 0  & 1 & 2 & 3 & 4 & 4 & 5 & 5 & 6 \\
0 & 8  & 0  & 1 & 2 & 3 & 4 & 5 & 5 & 6 & 6
\end{tabular}

\noindent
To find the actual LCS itself, start in the bottom right corner and traverse backwards through the table.
Following the path of the larger values until you reach the top or left edge of the table.
The elements in the sequence that correspond to the table cells you visit during this backtracking process will form the LCS.

\noindent
The traversal algorithm is as follows
\begin{enumerate}
    \item If the cell value directly to the left and directly above are the same as the current cell, go up and to the left by one.
    \item If this condition fails, go directly up.
\end{enumerate}
$\langle0,1,0,1,0,1\rangle$.
\end{solution}

\end{document}
