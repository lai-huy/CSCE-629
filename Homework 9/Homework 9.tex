\documentclass{article}
\usepackage{amsmath,amssymb,amsthm,latexsym,paralist,url}

\theoremstyle{definition}
\newtheorem{problem}{Problem}
\newtheorem*{solution}{Solution}
\newtheorem*{resources}{Resources}

\newcommand{\name}[1]{\noindent\textbf{Name:} {#1}}
\newcommand{\honor}{\noindent On my honor, as an Aggie, I have neither
  given nor received any unauthorized aid on any portion of the
  academic work included in this assignment. Furthermore, I have
  disclosed all resources (people, books, web sites, etc.) that have
  been used to prepare this homework. \\[1ex]
 \textbf{Signature:} \underline{\hspace*{5cm}} }
 
\newcommand{\checklist}{\noindent\textbf{Checklist:}
\begin{compactitem}[$\Box$] 
\item Did you add your name? 
\item Did you disclose all resources that you have used? \\
(This includes all people, books, websites, etc. that you have consulted)
\item Did you sign that you followed the Aggie honor code? 
\item Did you solve all problems? 
\item Did you submit the pdf file resulting from your latex file 
  of your homework?
\end{compactitem}
}

\newcommand{\problemset}[1]{\begin{center}\textbf{Problem Set #1}\end{center}}
\newcommand{\duedate}[1]{\begin{quote}\textbf{Due date:} Electronic
    submission of this homework is due on \textbf{#1} on canvas.\end{quote}}

\newcommand{\N}{\mathbf{N}}
\newcommand{\R}{\mathbf{R}}
\newcommand{\Z}{\mathbf{Z}}


\begin{document}
\problemset{9}
\duedate{4/14/2023}
\name{ (put your name here)}
\begin{resources} (All people, books, articles, web pages, etc. that
  have been consulted when producing your answers to this homework)
\end{resources}
\honor

\newpage

Read the chapter on NP-completeness in our textbook.  

\begin{problem} (30 points) 
Let DSAT denote the problem to decide whether a Boolean formula has
at least two satisfying assignments. Show that 
\begin{enumerate}[(a)]
\item DSAT is in NP
\item 3SAT $\le_p$ DSAT
\end{enumerate}
Conclude that DSAT is NP complete. [Hint: Introduce a new variable] 
\end{problem}
\begin{solution}
\end{solution}

\begin{problem} (20 points)
Theorem 34.11 in our textbook shows that CLIQUE is NP-complete using
the reduction 3SAT $\le_p$ CLIQUE. (a) Describe the graph corresponding to
the 3SAT instance 
$$ (x_1 \vee \neg x_2 \vee \neg x_3) \wedge (\neg x_1 \vee \neg x_2
\vee x_3) \wedge (\neg x_1 \vee x_2 \vee \neg x_3).
$$
in this reduction. (b) Find a satisfying assignment and the corresponding
clique in the graph. [Hint: The package tikz can be helpful in drawing
beautiful graphs in LaTeX.] 
\end{problem}
\begin{solution}
\end{solution}

\begin{problem} (30 points) 
The SUBSET SUM problem asks to decide whether a finite set S of
positive integers has a subset $T$ such that the elements of $T$ sum
to a positive integer $t$. (a) Is $(S,t)$ a yes-instance when the set $S$
is given by 
$$ S = \{ 2,3,5,7,8\} $$
and $t= 19$? Prove your result. (b) Why is a brute force algorithm not
feasible for larger sets $S$ (c) Explain in your own words why the 
dynamic programming
solution to SUBSET SUM given in 
\begin{center}
\url{https://www.cs.dartmouth.edu/~deepc/Courses/S19/lecs/lec6.pdf}
\end{center}
is not a polynomial time algorithm. 
\end{problem}
\begin{solution}
\end{solution}


\begin{problem} (20 points) % Set partition is NP complete
Exercise 34.5-5 on page 1101 using the reduction SUBSET SUM $\le_p$ SET PARTITION.
\end{problem}
\begin{solution}
\end{solution}

Make sure that you write the solutions in your own words!
\medskip



\goodbreak
\checklist
\end{document}
