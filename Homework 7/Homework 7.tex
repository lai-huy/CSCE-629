\documentclass{article}
\usepackage{amsmath,amssymb,amsthm,latexsym,paralist}
\usepackage{tikz}
\usepackage{minted}

\theoremstyle{definition}
\newtheorem{problem}{Problem}
\newtheorem*{solution}{Solution}
\newtheorem*{resources}{Resources}

\newcommand{\name}[1]{\noindent\textbf{Name: #1}}
\newcommand{\honor}{\noindent On my honor, as an Aggie, I have neither
  given nor received any unauthorized aid on any portion of the
  academic work included in this assignment. Furthermore, I have
  disclosed all resources (people, books, web sites, etc.) that have
  been used to prepare this homework. \\[1ex]
 \textbf{Signature:} \underline{\hspace*{5cm}} }

 \newcommand{\checklist}{\noindent\textbf{Checklist:}
\begin{compactitem}[$\Box$] 
\item Did you add your name? 
\item Did you disclose all resources that you have used? \\
(This includes all people, books, websites, etc. that you have consulted)
\item Did you sign that you followed the Aggie honor code? 
\item Did you solve all problems? 
\item Did you typeset your answers entirely in LaTeX? 
\item Did you submit the pdf file of your homework?
\end{compactitem}
}



\newcommand{\problemset}[1]{\begin{center}\textbf{Problem Set
      #1}\end{center}}
\newcommand{\duedate}[2]{\begin{quote}\textbf{Due dates:} Electronic
    submission of the pdf file of this homework is due on
    \textbf{#1} on canvas. The homework must be typeset with LaTeX to
    receive any credit. All answers must be formulated in your own words.\end{quote} }

\newcommand{\N}{\mathbf{N}}
\newcommand{\R}{\mathbf{R}}
\newcommand{\Z}{\mathbf{Z}}


\begin{document}
\problemset{7}
\duedate{3/24/2023 before 11:59pm}

\textbf{Watch out for additional material that will appear on
  Thursday! Deadline is on Friday, as usual.} 

\name{ Huy Quang Lai}
\begin{resources} Cormen, Thomas H., Leiserson, Charles E., Rivest, Ronald L., Stein, Clifford. \textit{Introduction to Algorithms}. The MIT Press.
\end{resources}
\honor

\newpage

Read the chapters on ``Elementary Graph Algorithms'' and
``Single-Source Shortest Paths'' in our textbook before
attempting to answer these questions.

\begin{problem}[20 points]
Give an algorithm that determines whether or not a given undirected
graph $G=(V,E)$ contains a cycle. Your algorithm should run in $O(V)$
time, independent of the number $|E|$ of edges. 
\end{problem}
\begin{solution}
\end{solution}

\noindent
An undirected graph is acyclic if and only if DFS yields no back edges 

\begin{minted}{cpp}
bool DFS(size_t v, size_t parent, 
const vector<vector<size_t>>& graph, vector<bool>& visited) {
    visited[v] = true;
    for (size_t u : graph[v]) {
        if (!visited[u]) {
            if (DFS(u, v, graph, visited))
                return true;
        }
        else if (u != parent)
            return true;
    }
    return false;
}

bool has_cycle(const vector<vector<int>>& graph) {
    size_t n = graph.size();
    vector<bool> visited(n, false);
    for (size_t v = 0; v < n; ++v)
        if (!visited[v])
            if (DFS(v, -1, graph, visited))
                return true;
    return false;
}
\end{minted}

\begin{problem}[20 points]
Given a weighted, directed graph $G=(V,E)$ with no negative-weight
cycles, let $m$ be the maximum over all vertices $v\in V$ of the minimum
number of edges in a shortest path from the source $s$ to $v$. (Here, the
shortest path is by weight, not the number of edges.) Suggest a simple
change to the Bellman-Ford algorithm that allows it to terminate in $m+1$
passes, even if $m$ is not known in advance.
\end{problem}
\begin{solution}
\end{solution}

\noindent
By the upper bound theory, we know that after $m$ iterations, no $d$ values will ever change.
Therefore, under induction, no $d$ values will change in the $(m+1)$-th iteration.  
However, since the $m$ value is not knowsn in advance, the algorithm cannot iteract exactly $m$ times and then terminate.

\begin{problem}[20 points]
Suppose that we change line 4 of Dijkstra’s algorithm in our textbook
to the following.\\[1ex]
4 \textbf{while} $|Q| > 1$ \\[1ex] 
This change causes the while loop to execute $|V|-1$ times instead of
$|V |$ times. Is this proposed algorithm correct?
[Use the version of Dijkstra's algorithm from the textbook]
\end{problem}
\begin{solution}
This proposed change will not provide a correct output.

\noindent
Dijkstra's algorithm relies on visiting all vertices in the graph to ensure that the shortest path to each vertex is found. By changing the while loop condition to $|Q|>1$, the algorithm will terminate early before all vertices have been visited.\\

\noindent
Consider a scenario where the graph has $n$ vertices and the shortest path from the source vertex to the destination vertex requires visiting all $n$ vertices. With the modified while loop condition, the algorithm would terminate after $n-1$ iterations, leaving the shortest path undiscovered.\\

\noindent
Therefore, it is essential to have the while loop execute $|V|$ times to ensure that all vertices are visited, and the algorithm finds the correct shortest paths.
\end{solution}

\clearpage
\begin{problem}[40 points]
  Help Professor Charlie Eppes find the most likely escape routes of
  thieves that robbed a bookstore on Texas Avenue in College
  Station. The map will be published on Thursday evening. In preparation, you
  might want to implement Dijkstra's single-source shortest path
  algorithm, so that you can join the manhunt on Thursday
  evening. [Edge weight 1 means very desirable street, weight 2 means
  less desirable street] 
\end{problem}
\begin{solution}
Destination Nodes, 6, 8, 9, 15, 16, 22\\
Since the graph is undirected a connection exists both ways.
The format of the graph is \verb+source <->{destination} weight+.\\
\begin{verbatim}
1 <->{2} 1, 1 <->{11} 1, 2 <->{3} 1, 2 <->{21} 1, 3 <->{4} 1
3 <->{8} 2, 4 <->{5} 1, 5 <->{6} 2, 5 <->{7} 1, 6 <->{7} 1
7 <->{8} 1, 8 <->{9} 1, 9 <->{10} 1, 9 <->{19} 1
10 <->{11} 1, 10 <->{18} 1, 11 <->{12} 2, 11 <->{17} 1
12 <->{13} 2, 13 <->{14} 2, 13 <->{21} 1, 14 <->{15} 1
14 <->{16} 1, 14 <->{20} 1, 16 <->{17} 1, 17 <->{18} 1
18 <->{19} 2, 20 <->{21} 2, 20 <->{22} 1, 21 <->{22} 2
\end{verbatim}

\noindent
Using Dijkstra's algorithm, results in the following shortest paths to the Destination Nodes.
\begin{verbatim}
1 --> 2 --> 3 --> 4 --> 5 --> 6 distance: 6
1 --> 2 --> 3 --> 8 distance: 4
1 --> 11 --> 10 --> 9 distance: 3
1 --> 11 --> 17 --> 16 --> 14 --> 15 distance: 5
1 --> 11 --> 17 --> 16 distance: 3
1 --> 2 --> 21 --> 22 distance: 4
\end{verbatim}
\end{solution}
\end{document}
