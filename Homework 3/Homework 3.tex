\documentclass{article}
\usepackage{amsmath,amssymb,amsthm,latexsym,paralist}

\theoremstyle{definition}
\newtheorem{problem}{Problem}
\newtheorem*{solution}{Solution}
\newtheorem*{resources}{Resources}

\newcommand{\name}[1]{\noindent\textbf{Name: #1}}
\newcommand{\honor}{\noindent On my honor, as an Aggie, I have neither
  given nor received any unauthorized aid on any portion of the
  academic work included in this assignment. Furthermore, I have
  disclosed all resources (people, books, web sites, etc.) that have
  been used to prepare this homework. \\[1ex]
 \textbf{Signature:} \underline{\hspace*{5cm}} }

 \newcommand{\checklist}{\noindent\textbf{Checklist:}
\begin{compactitem}[$\Box$] 
\item Did you add your name? 
\item Did you disclose all resources that you have used? \\
(This includes all people, books, websites, etc. that you have consulted)
\item Did you sign that you followed the Aggie honor code? 
\item Did you solve all problems? 
\item Did you submit the pdf file of your homework? Check!
\end{compactitem}
}



\newcommand{\problemset}[1]{\begin{center}\textbf{Problem Set
      #1}\end{center}}
\newcommand{\duedate}[1]{\begin{quote}\textbf{Due date:} Electronic
    submission of the pdf file of this homework is due on
    \textbf{#1} on ecampus. \end{quote} }

\newcommand{\N}{\mathbf{N}}
\newcommand{\R}{\mathbf{R}}
\newcommand{\Z}{\mathbf{Z}}


\begin{document}
\problemset{3}
\duedate{2/10/2023 before 11:59pm}
\name{ Huy Quang Lai }
\begin{resources} (All people, books, articles, web pages, etc. that
  have been consulted when producing your answers to this homework)
\end{resources}
\honor

\newpage
\begin{problem}[20 points]
  Use mathematical induction to show that when $n$ is an exact power of 2, the solution of the recurrence
$$ T(n) = \begin{cases}
2 & \text{if $n=2$,} \\
2T(n/2) + n & \text{if $n=2^k$ for $k>1$,}
\end{cases}
$$
is $T(n)=n\log_2 n$. 
\end{problem}
\begin{solution}
\textbf{Base Case}:\\
Let $k=1$\\
$T(2)=2$\\
$2\log_2 2=2$\\
The relationship holds when $n=2$\\

\noindent
\textbf{Inductive Case}:\\
Assume that $T(n)=n\log_2 n$ if $n=2^{k}, \forall k>1$\\
If $n=2^{k+1}$, then:
\begin{flalign*}
T(2^{k+1})  &= 2T(2^{k+1}/2)+2^{k+1}    &\\
            &= 2T(2^k)+2^{k+1}          &\\
            &= 2\left(2^k\log_2\left(2^k\right)\right)+2^{k+1} & \text{Inductive Hypothesis}\\
            &= 2^{k+1}\log_2(2^k)+2^{k+1}   &\\
            &= 2^{k+1}\left(\log_2(2^k)+1\right)    &\\
            &= 2^{k+1}\log_2(2^{k+1})
\end{flalign*}

\noindent
By the inductive hypothesis, $T(n)\equiv n\log_2 n$\\
$\qed$
\end{solution}

\begin{problem}[20 points]
  We can express insertion sort as a recursive procedure as follows. 
In order to sort $A[1..n]$, we recursively sort $A[1..n-1]$ and then
insert $A[n]$ into the sorted array $A[1..n-1]$. Write a recurrence for
the running time of this recursive version of insertion sort. 
\end{problem}
\begin{solution}
There are two steps in the algorithm.
\begin{enumerate}
    \item sort the sub-array $A[1..n-1]$
    \item inserting $A[n]$ into the sorted sub-array.
\end{enumerate}
For $n=1$, step 1 does not take time since the sub-array is empty. Step 2 takes constant time.\\
As a result, the algorithm runs in $\Theta(1)$.

\noindent
For $n>1$, step 1 calls for the sorting of $A[1..n-1]$. Step 2 takes $\Theta(n)$ since, on average, $A[n]$ will be inserted in the middle of the array.

\[T(n)=\begin{cases}
\Theta(1) & \text{if $n=1$.}\\
T(n-1)+\Theta(n) & \text{if $n>1$.}
\end{cases}\]
\end{solution}

\clearpage
\begin{problem}[20 points]
  V. Pan has discovered a way of multiplying $68 \times 68$ matrices
  using $132,464$ multiplications, a way of multiplying $70\times 70$
  matrices using $143,640$ multiplications, and a way of multiplying
  $72\times 72$ matrices using $155,424$ multiplications. Which method
  yields the best asymptotic running time when used in a
  divide-and-conquer matrix-multiplication algorithm? How does it
  compare to Strassen’s algorithm?
\end{problem}
\begin{solution}
For $m$ sub-problems and $k$ multiplications, the recurrence relationship is as follows
\[T(n)=kT\left(\frac{n}{m}\right)\]

\noindent
Using the master theorem, $T(n)=\Theta\left(n^{\log_{m}k}\right)$\\

\noindent
$\log_{68}(132464)\approx2.79512848736$\\
$\log_{70}(143640)\approx2.79512268975$\\
$\log_{72}(155424)\approx2.79514739109$\\
$\log_2{7}\approx2.80735492206$\\
$\because\log_2{7}>\log_{72}(155424),$\\
$\therefore$ This algorithm runs asymptotically faster than Stranssen's Algorithm
\end{solution}

\begin{problem}[20 points]
  Show how to multiply the complex numbers $a+bi$ and $c+di$ using
  only three multiplications of real numbers. The algorithm should
  take $a$, $b$, $c$, and $d$ as input and produce the real component
  $ac-bd$ and the imaginary component $ad+bc$ separately.
\end{problem}
\begin{solution}
Using Karatsuba's Algorithm, the three multiplications for calulcate the product would be
\begin{enumerate}
    \item $S_0=ac$
    \item $S_1=bd$
    \item $S_3=(a+b)(c+d)$
\end{enumerate}
Using these three products, the final product can be determined as follows.
\[A=S_1-S_2,B=S_3-(S_1+S_2)\]

\end{solution}

\begin{problem}[20 points]
Use the master method to show that the solution to the binary-search 
recurrence 
$$ T(n) = T(n/2)+\Theta(1)$$
is $T(n) = \Theta(\lg n)$. Clearly indicate which case of the Master
theorem is used.  
\end{problem}
\begin{solution}
$a=1,b=2,f(n)=\Theta(1)$\\
$\because f(n)=\Theta\left(n^{\log_b a}\right)$\\
$\therefore$ Case II of the Master Theorem is used.
\[T(n)=\Theta\left(n^{\log_b a}\log_2 n\right)=\Theta(\log_2 n)\]
\end{solution}
\end{document}
