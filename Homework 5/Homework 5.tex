\documentclass{article}
\usepackage{amsmath,amssymb,amsthm,latexsym,paralist}

\theoremstyle{definition}
\newtheorem{problem}{Problem}
\newtheorem*{solution}{Solution}
\newtheorem*{resources}{Resources}

\newcommand{\name}[1]{\noindent\textbf{Name: #1}}
\newcommand{\honor}{\noindent On my honor, as an Aggie, I have neither
  given nor received any unauthorized aid on any portion of the
  academic work included in this assignment. Furthermore, I have
  disclosed all resources (people, books, web sites, etc.) that have
  been used to prepare this homework. \\[1ex]
 \textbf{Signature:} \underline{\hspace*{5cm}} }

\newcommand{\checklist}{\noindent\textbf{Checklist:}
\begin{compactitem}[$\Box$] 
\item Did you add your name? 
\item Did you disclose all resources that you have used? \\
(This includes all people, books, websites, etc. that you have consulted)
\item Did you sign that you followed the Aggie honor code? 
\item Did you solve all problems? 
\item Did you submit the pdf file of your homework?
\end{compactitem}
}



\newcommand{\problemset}[1]{\begin{center}\textbf{Problem Set
      #1}\end{center}}
\newcommand{\duedate}[1]{\begin{quote}\textbf{Due date:} Electronic
    submission of the pdf file of this homework is due on
    \textbf{#1} on canvas. \end{quote} }

\newcommand{\N}{\mathbf{N}}
\newcommand{\R}{\mathbf{R}}
\newcommand{\Z}{\mathbf{Z}}


\begin{document}
\problemset{5}
\duedate{2/24/2023 before 11:59pm}
\name{ (put your name here)}
\begin{resources} (All people, books, articles, web pages, etc. that
  have been consulted when producing your answers to this homework)
\end{resources}
\honor

\newpage
Make sure that you describe all solutions in your own
words. Typesetting in \LaTeX{} is required. Read
chapters 15 and 16 in our textbook. 

\begin{problem}[20 points]
  Find an optimal parenthesization of a matrix-chain product whose
  sequence of dimensions is $\langle 5, 10, 3, 12, 5, 50, 6\rangle$. 
  Explain how you found the solution.
\end{problem}
\begin{solution}
\end{solution}

\begin{problem}[20 points]
  Use a proof by induction to show that the solution to the
  recurrence
  $$ P(n) = 
  \begin{cases}
    1& \text{if $n=1$,} \\
    \sum_{k=1}^{n-1} P(k)P(n-k) & \text{if $n\ge 2$.}
  \end{cases}
  $$
    is $\Omega(2^n)$. 
\end{problem}
\begin{solution}
\end{solution}

\begin{problem}[20 points]
Let $R(i,j)$ be the number of times that table entry $m[i,j]$ is
referenced while computing other table entries in a call of
MATRIX-CHAIN-ORDER. Show that the total number of references for the
entire table is
$$ \sum_{i=1}^n\sum_{j=i}^n R(i,j) = \frac{n^3-n}{3}.$$
\end{problem}
\begin{solution}
\end{solution}

\begin{problem}[20 points]
  Give an $O(n^2)$-time algorithm to find the longest monotonically
  increasing subsequence of a sequence of n numbers.
\end{problem}
\begin{solution}
\end{solution}

\begin{problem}[20 points]
  Describe an efficient algorithm that, given a set
  $$\{ x_1, x_2, \ldots, x_n\}$$ of $n$ points on the real line,
  determines the smallest set of unit-length closed intervals that
  contains all of the given points. Argue that your algorithm is
  correct.
\end{problem}
\begin{solution}
\end{solution}


Discussions on canvas are always encouraged, especially to clarify
concepts that were introduced in the lecture. However, discussions of
homework problems on canvas should not contain spoilers. It is okay to
ask for clarifications concerning homework questions if needed. Make
sure that you write the solutions in your own words. 


\medskip



\goodbreak
\checklist
\end{document}
