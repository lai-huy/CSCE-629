\documentclass{article}
\usepackage{amsmath,amssymb,amsthm,latexsym,paralist}
\usepackage{minted}

\theoremstyle{definition}
\newtheorem{problem}{Problem}
\newtheorem*{solution}{Solution}
\newtheorem*{resources}{Resources}

\newcommand{\name}[1]{\noindent\textbf{Name: #1}}
\newcommand{\honor}{\noindent On my honor, as an Aggie, I have neither
  given nor received any unauthorized aid on any portion of the
  academic work included in this assignment. Furthermore, I have
  disclosed all resources (people, books, web sites, etc.) that have
  been used to prepare this homework. \\[1ex]
 \textbf{Signature:} \underline{\hspace*{5cm}} }

\newcommand{\checklist}{\noindent\textbf{Checklist:}
\begin{compactitem}[$\Box$] 
\item Did you add your name? 
\item Did you disclose all resources that you have used? \\
(This includes all people, books, websites, etc. that you have consulted)
\item Did you sign that you followed the Aggie honor code? 
\item Did you solve all problems? 
\item Did you submit the pdf file of your homework?
\end{compactitem}
}



\newcommand{\problemset}[1]{\begin{center}\textbf{Problem Set
      #1}\end{center}}
\newcommand{\duedate}[1]{\begin{quote}\textbf{Due date:} Electronic
    submission of the pdf file of this homework is due on
    \textbf{#1} on canvas. \end{quote} }

\newcommand{\N}{\mathbf{N}}
\newcommand{\R}{\mathbf{R}}
\newcommand{\Z}{\mathbf{Z}}


\begin{document}
\problemset{5}
\duedate{2/24/2023 before 11:59pm}
\name{Huy Quang Lai}
\begin{resources} Cormen, Thomas H., Leiserson, Charles E., Rivest, Ronald L., Stein, Clifford. \textit{Introduction to Algorithms}. The MIT Press.
\end{resources}
\honor

\newpage
Make sure that you describe all solutions in your own
words. Typesetting in \LaTeX{} is required. Read
chapters 15 and 16 in our textbook. 

\begin{problem}[20 points]
  Find an optimal parenthesization of a matrix-chain product whose
  sequence of dimensions is $\langle 5, 10, 3, 12, 5, 50, 6\rangle$. 
  Explain how you found the solution.
\end{problem}
\begin{solution}
Let $A=5\times10$ matrix,
$B=10\times3$ matrix,
$C=3\times12$ matrix,
$D=12\times5$ matrix,
$E=5\times50$ matrix,
$F=50\times6$ matrix

\noindent
The most optimal parenthesization of the matrix chain is
\[(AB)\left((CD)(EF)\right)\]
\end{solution}

\noindent
With this order, the three largest dimensions of 10, 12, and 50 are absorbed before they can make any other matrix multiplication take more operations. With this approach, we get the optimal 2010 operations.

\noindent
$AB$ is a $5\times3$ matrix that requires $5\times10\times3=150$ operations to compute.\\
$CD$ is a $3\times5$ matrix that requires $3\times12\times5=180$ operations to compute.\\
$EF$ is a $5\times6$ matrix that requires $5\times50\times6=1500$ operations to compute.

\noindent
From here, there is a choice to group $(AB)(CD)$ or $(CD)(EF)$\\
$(AB)(CD)$ is a $5\times5$ matrix that requires an additional $5\times3\times5=75$ operations to compute.\\
$(CD)(EF)$ is a $3\times6$ matrix that requires an additional $3\times5\times6=90$ operations to compute.

\noindent
After this grouping a final multiplication is needed.\\
$(AB)((CD)(EF))$ is a $5\times6$ matrix that requires an additional $5\times3\times6=90$ operations to compute.\\
$((AB)(CD))(EF)$ is a $5\times6$ matrix that requires an additional $5\times5\times6=150$ operations to compute.

\noindent
Although more locally optimal to group $(AB)(CD)$, the grouping of $(CD)(EF)$ is more globally optimal.

\clearpage
\begin{problem}[20 points]
  Use a proof by induction to show that the solution to the
  recurrence
  $$ P(n) = 
  \begin{cases}
    1& \text{if $n=1$,} \\
    \sum_{k=1}^{n-1} P(k)P(n-k) & \text{if $n\ge 2$.}
  \end{cases}
  $$
    is $\Omega(2^n)$. 
\end{problem}
\begin{solution}
Base Case:\\
Let $n=2$
\begin{flalign*}
P(2)    &= \sum_{k=1}^{2-1}P(k)P(2-k)   &\\
        &= P(1)P(1)                     &\\
        &= 1 
\end{flalign*}
$\because1\geq2^{2-2}=2^{0}=1$\\
$P(n)\geq2^{n-2}$ holds for $n=2$

\noindent
Inductive Case:\\
Assume that $P(n)\geq2^{n-2}$
\begin{flalign*}
P(n)    &=\sum_{k=1}^{n-1}P(k)P(n-k)                                            &\\
        &=P(1)P(n-1)+\sum_{k=2}^{n-2}P(k)P(n-k)+P(n-1)P(1)                      &\\
        &\geq2^{n-3}+\sum_{k=2}^{n-2}\left(2^{k-2}2^{n-k-2}\right)+2^{n-3}      &\\
        &=2^{n-2}+\sum_{k=2}^{n-2}2^{n-4}                                       &\\
        &=2^{n-2}+2^{n-4}\sum_{k=2}^{n-2}1                                      &\\
        &=2^{n-2}+(n-3)\left(2^{n-4}\right)                                     &\\
        &=(n+1)2^{n-4}
\end{flalign*}
$\therefore P(n)\in\Omega(2^n)$\\
$\qed   $
\end{solution}

\clearpage
\begin{problem}[20 points]
Let $R(i,j)$ be the number of times that table entry $m[i,j]$ is
referenced while computing other table entries in a call of
MATRIX-CHAIN-ORDER. Show that the total number of references for the
entire table is
\[\sum_{i=1}^n\sum_{j=i}^n R(i,j) = \frac{n^3-n}{3}.\]
\end{problem}
\begin{solution}
From the textbook
\begin{flalign*}
\sum_{i=1}^n\sum_{j=i}^n R(i,j) &= \sum_{l=2}^{n}2(n-l+1)(l-1)                  &\\
                                &= 2\sum_{l=1}^{n-1}l(n-l)                      &\\
                                &= 2\sum_{l=1}^{n-1}nl-2\sum_{l=1}^{n-1}l^2     &\\
                                &= 2n\frac{n(n-1)}{2}-2\frac{n(n-2)(2n-1)}{6}   &\\
                                &= n^3-n^2-\frac{2n^3-3n^2+n}{3}                &\\
                                &= \frac{n^3-n}{3}
\end{flalign*}
$\qed$
\end{solution}

\clearpage
\begin{problem}[20 points]
  Give an $O(n^2)$-time algorithm to find the longest monotonically
  increasing subsequence of a sequence of n numbers.
\end{problem}
\begin{solution}
\[\]
\begin{minted}{python}
def longSeq(seq):
    n = len(seq)
    L = [1 for i in range(n)]
    for i in range(1, n):
        for j in range(0, i):
            if seq[j] < seq[i]:
                L[i] = max(L[j] + 1, L[i])
                
    max_length = max(L)
    current_length = max_length
    subseq = []
    
    for i in range(n - 1, -1, -1):
        if L[i] == current_length:
            current_length -= 1
            subseq.append(seq[i])
    subseq.reverse()
    return subseq

\end{minted}
\end{solution}

\clearpage
\begin{problem}[20 points]
  Describe an efficient algorithm that, given a set
  $$\{ x_1, x_2, \ldots, x_n\}$$ of $n$ points on the real line,
  determines the smallest set of unit-length closed intervals that
  contains all of the given points. Argue that your algorithm is
  correct.
\end{problem}
\begin{solution}
First, sort the set of $n$ points $\{ x_1, x_2, \ldots, x_n\}$ to get the set $Y=\{y_1, y_2, \cdots, y_n\}$ such that $y_1\leq y_2\leq\cdots\leq y_n$.\\
Next, linearly scan over $Y$. For every $y_i\in Y$, place the closed interval $[y_i,y_i+1]$ in the optimal solution set. Call this set $S$.\\
Remove all the points in $Y$ that are covered by $[y_i,y_i+1]$.\\
Repeat this process for every point $y_i$ until $Y$ is empty.

\noindent
$S$ should be the optimal solution.\\
Suppose that there exists an optimal solution $S^*$ such that $y_1$ is covered by $[x',x'+1]\in S^*$ where $x'<1$.\\
Since $y_1$ is the left most element of the given set, there is no other point lying in $[x',y_1)$.\\
Therefore, replacing $[x',x'+1]$ in $S^*$ with $[y_1,y_1+1]$ will result in another optimal solution.\\
Since this idea can be applied to every point on $Y$, $S$ is the optimal solution.

\noindent
The runtime of the algorithm is $O(n\log n)$ due to the initial sort. The linear traversal of the set is $O(n)$. With this the total runtime of the algorithm is
\[O(n\log n)\]
\end{solution}
\end{document}
