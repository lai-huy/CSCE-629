\documentclass{article}
\usepackage{amsmath,amssymb,amsthm,latexsym,paralist}

\DeclareRobustCommand{\stirling}{\genfrac\{\}{0pt}{}}

\theoremstyle{definition}
\newtheorem{problem}{Problem}
\newtheorem*{solution}{Solution}
\newtheorem*{resources}{Resources}

\newcommand{\name}[1]{\noindent\textbf{Name: #1}}
\newcommand{\honor}{\noindent On my honor, as an Aggie, I have neither
  given nor received any unauthorized aid on any portion of the
  academic work included in this assignment. Furthermore, I have
  disclosed all resources (people, books, web sites, etc.) that have
  been used to prepare this homework. The work shown here is entirely
  my own, and is written in my own words.\\[1ex]
 \textbf{Signature:} \underline{\hspace*{5cm}} }

\newcommand{\problemset}[1]{\begin{center}\textbf{Problem Set #1}\end{center}}
\newcommand{\duedate}[1]{\begin{quote}\textbf{Due date:} Electronic
    submission the .pdf file of this homework is due on \textbf{#1} on canvas
    (as a turnitin assignment).\end{quote}}

\newcommand{\N}{\mathbf{N}}
\newcommand{\R}{\mathbf{R}}
\newcommand{\Z}{\mathbf{Z}}

\begin{document}
\problemset{1}
\centerline{CSCE 411/629 (Dr. Klappenecker) }

\duedate{Friday, 1/27, 11:59pm}
\name{ (Huy Quang Lai)}
\begin{resources}
Perusall\\
Cormen, Thomas H., Leiserson, Charles E., Rivest, Ronald L., Stein, Clifford. \textit{Introduction to Algorithms}. The MIT Press.
\end{resources}
\honor

\newpage%
\noindent Get familiar with \LaTeX. All exercises in this homework are from the lecture
notes on perusall, not from our textbook. \medskip

\noindent\textbf{Reading assignment:} Carefully read the lecture notes
\verb|dm_ch11.pdf| on Perusall. Skim Appendices A and B in the textbook. 

\begin{problem}
Exercise 11.3 (in notes on perusall)\\
Let $f(n)=n^2+2n,g(n)=n^2$\\
Is $f\sim g$?
\begin{solution}
$\displaystyle\lim_{n\to\infty}\frac{f(n)}{g(n)}=\lim_{n\to\infty}\frac{n^2+2n}{n^2}=\lim_{n\to\infty}\left(1+\frac{1}{n}\right)=1$

\noindent
$\displaystyle\because\lim_{n\to\infty}\frac{f(n)}{g(n)}=1$\\
$\therefore$ Ernie is correct.
\end{solution}
\end{problem}

\begin{problem}
Exercise 11.9 (in notes on perusall)
\begin{enumerate}[(a)]
    \item $\displaystyle f(n)=(-1)^n$
    \item $\displaystyle f(n)=4+\frac{(-1)^nn}{n+10}$
    \item $\displaystyle f(n)=\left((-1)^n+(-1)^{\left\lfloor\frac{n}{2}\right\rfloor}\right)\left(1+\frac{1}{n}\right)$
\end{enumerate}
\begin{solution}
\begin{enumerate}[(a)]
    \item When $n$ is even, $f(n)=1$. Similarly, when $n$ is odd, $f(n)=-1$.\\
    $\therefore$ the accumulation points for $f(n)$ are $-1$, the lower accumulation point, and $1$, the upper accumulation point.

    \item $\displaystyle\because\lim_{n\to\infty}\left(\frac{n}{n+10}\right)=1$\\
    $\displaystyle\therefore\lim_{n\to\infty}f(n)=\lim_{n\to\infty}(4+(-1)^n)$.\\
    With this result in combination with the result from 2.a, there are two accumulation points. $3$ is the lower accumulation point and $5$ is the upper accumulation point.

    \clearpage
    \item There are four different sub-sequences that exhibit different behaviors.
    \begin{flalign*}
    f(4n)   &=\left((-1)^{4n}+(-1)^{\left\lfloor\frac{4n}{2}\right\rfloor}\right)\left(1+\frac{1}{4n}\right) &\\
            &=\left(1+(-1)^{2n}\right)\left(1+\frac{1}{4n}\right) &\\
            &=2\left(1+\frac{1}{4n}\right)
    \end{flalign*}

    \begin{flalign*}
    f(4n+1) &=\left((-1)^{4n+1}+(-1)^{\left\lfloor\frac{4n+1}{2}\right\rfloor}\right)\left(1+\frac{1}{4n+1}\right) &\\
            &=\left(-1+(-1)^{2n}\right)\left(1+\frac{1}{4n+1}\right) &\\
            &=0\left(1+\frac{1}{4n+1}\right) &\\ &=0
    \end{flalign*}

    \begin{flalign*}
    f(4n+2) &=\left((-1)^{4n+2}+(-1)^{\left\lfloor\frac{4n+2}{2}\right\rfloor}\right)\left(1+\frac{1}{4n+2}\right) &\\
            &=\left(1+(-1)^{2n+1}\right)\left(1+\frac{1}{4n+2}\right) &\\
            &=\left(1-1\right)\left(1+\frac{1}{4n+2}\right) &\\
            &=0\left(1+\frac{1}{4n+2}\right) &\\ &=0
    \end{flalign*}

    \begin{flalign*}
    f(4n+3) &=\left((-1)^{4n+3}+(-1)^{\left\lfloor\frac{4n+3}{2}\right\rfloor}\right)\left(1+\frac{1}{4n+3}\right) &\\
            &=\left(-1+(-1)^{2n+1}\right)\left(1+\frac{1}{4n+3}\right) &\\
            &=\left(-1-1\right)\left(1+\frac{1}{4n+3}\right) &\\
            &=-2\left(1+\frac{1}{4n+3}\right)
    \end{flalign*}

    Both $f(4n+1)$ and $f(4n+2)$ give an accumulation point of $0$.\\
    $f(4n)$ yields an accumulation point of $2$ while $f(4n+3)$ yields an accumulation point of $-2$.\\
    $\therefore-2$ is a lower accumulation point and $2$ is an upper accumulation point.
\end{enumerate}
\end{solution}
\end{problem}

\clearpage
\begin{problem}
Exercise 11.17 (in notes on perusall)\\
Let $b$ and $d$ be positive real numbers that are not equal to 1.
\begin{enumerate}[(a)]
    \item Show that $\displaystyle\Theta(\log_b n)=\Theta(\log_d n)$
    \item Does $\displaystyle\Theta\left(n^{\log_b n}\right)=\Theta\left(n^{\log_d n}\right)$ hold in general?
\end{enumerate}
\begin{solution}
\begin{enumerate}[(a)]
    \item Using the change of base rule for logarithms, this yeilds:
    \[\log_b n=\frac{\log_d n}{\log_d b}\]
    $\because\log_d b$ is a constant\\
    $\therefore\Theta(\log_b n)=\Theta(\log_d n)$

    \item The equality only holds if $b=d$\\
    If $b>d>1$, then
    \[\limsup_{n\to\infty}\frac{n^{\log_b n}}{n^{\log_d n}}=+\infty\]
    
    $\therefore n^{\log_b n}\not\in \Theta(\log_d n)$.\\
    This result leads to:\\
    $\Theta\left(n^{\log_b n}\right)\neq\Theta\left(n^{\log_d n}\right)$\\
    $\qed$
\end{enumerate}
\end{solution}
\end{problem}

\begin{problem}
Exercise 11.19 (in notes on perusall)\\
Show that for all positive integers $k$, we have
\[1^k+2^k+\cdots+n^k=\Theta(n^{k+1})\]
\begin{solution}
\end{solution}
For all positive integers $k$, the function $x^k$ is increasing.\\
Therefore,
\[\frac{1}{k+1}x^{k+1}=\int_0^nx^kdx\leq\sum_{m=1}^nm^k\]
holds for all $n\geq1$
\end{problem}

\begin{problem}
Exercise 11.34 (in notes on perusall)\\
Prove or disprove: $\displaystyle n^{\ln n}\in O\left(e^{\ln^2n}\right)$
\begin{solution}
\begin{flalign*}
n^{\ln n}   &=e^{\ln(n^{\ln n})}    &\\
            &=e^{\ln^2n}            &
\end{flalign*}
$\therefore n^{\ln n}\in O\left(e^{\ln^2n}\right)$\\
$\qed$
\end{solution}
\end{problem}
\end{document}
