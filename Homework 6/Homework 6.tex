\documentclass{article}
\usepackage{amsmath,amssymb,amsthm,latexsym,paralist}

\theoremstyle{definition}
\newtheorem{problem}{Problem}
\newtheorem*{solution}{Solution}
\newtheorem*{resources}{Resources}

\newcommand{\name}[1]{\noindent\textbf{Name: #1}}
\newcommand{\honor}{\noindent On my honor, as an Aggie, I have neither
  given nor received any unauthorized aid on any portion of the
  academic work included in this assignment. This work is my own. Furthermore, I have
  disclosed all resources (people, books, web sites, etc.) that have
  been used to prepare this homework. Looking up solutions is prohibited. \\[1ex]
 \textbf{Signature:} \underline{\hspace*{5cm}} }

\newcommand{\checklist}{\noindent\textbf{Checklist:}
\begin{compactitem}[$\Box$] 
\item Did you add your name? 
\item Did you disclose all resources that you have used? \\
(This includes all people, books, websites, etc. that you have consulted)
\item Did you sign that you followed the Aggie honor code? 
\item Did you solve all problems? 
\item Did you submit the pdf file of your homework?
\end{compactitem}
}



\newcommand{\problemset}[1]{\begin{center}\textbf{Problem Set
      #1}\end{center}}
\newcommand{\duedate}[1]{\begin{quote}\textbf{Due date:} Electronic
    submission of the pdf file of this homework is due on
    \textbf{#1} on canvas. \end{quote} }

\newcommand{\N}{\mathbf{N}}
\newcommand{\R}{\mathbf{R}}
\newcommand{\Z}{\mathbf{Z}}


\begin{document}
\problemset{6}
\duedate{3/10/2023 before 11:59pm}
\name{ (put your name here)}
\begin{resources} (All people, books, articles, web pages, etc. that
  have been consulted when producing your answers to this homework)
\end{resources}
\honor

\newpage
\textbf{Make sure that you describe all solutions in your own
words, and that the work presented is your own!} Typesetting in
\LaTeX{} is required. Read the chapter on amortized analysis in our textbook. 

\begin{problem}[20 points] In the lecture, we discussed a stack with
  PUSH, POP, and MULTIPOP operations.  If the set of stack operations
  included a MULTIPUSH operation, which pushes $k$ items onto the
  stack, would the $O(1)$ bound on the amortized cost of stack
  operations continue to hold?
\end{problem}
\begin{solution}
\end{solution}

\begin{problem}[20 points]
  In the lecture, we discussed a $k$-bit counter with an INCREMENT
  operation.  Show that if a DECREMENT operation were included in the $k$-bit
  counter example, $n$ operations could cost as much as $\Theta(nk)$
  time.
\end{problem}
\begin{solution}
\end{solution}

\begin{problem}[20 points] \label{base}
  Suppose we perform a sequence of $n$ operations on a data structure
  in which the $i$-th operation costs $i$ if $i$ is an exact power of
  $2$, and $1$ otherwise. Use aggregate analysis to determine the
  amortized cost per operation.
\end{problem}
\begin{solution}
\end{solution}

\begin{problem}[20 points]
 Redo Problem~\ref{base} using an accounting method of analysis.
\end{problem}
\begin{solution}
\end{solution}

\begin{problem}[20 points]
Redo Problem~\ref{base} using a potential method of analysis.
\end{problem}
\begin{solution}
\end{solution}


Discussions on canvas are always encouraged, especially to clarify
concepts that were introduced in the lecture. However, discussions of
homework problems on canvas should not contain spoilers. It is okay to
ask for clarifications concerning homework questions if needed. Make
sure that you write the solutions in your own words. 


\medskip



\goodbreak
\checklist
\end{document}
