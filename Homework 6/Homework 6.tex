\documentclass{article}
\usepackage{amsmath,amssymb,amsthm,latexsym,paralist}

\theoremstyle{definition}
\newtheorem{problem}{Problem}
\newtheorem*{solution}{Solution}
\newtheorem*{resources}{Resources}

\newcommand{\name}[1]{\noindent\textbf{Name: #1}}
\newcommand{\honor}{\noindent On my honor, as an Aggie, I have neither
  given nor received any unauthorized aid on any portion of the
  academic work included in this assignment. This work is my own. Furthermore, I have
  disclosed all resources (people, books, web sites, etc.) that have
  been used to prepare this homework. Looking up solutions is prohibited. \\[1ex]
 \textbf{Signature:} \underline{\hspace*{5cm}} }

\newcommand{\checklist}{\noindent\textbf{Checklist:}
\begin{compactitem}[$\Box$] 
\item Did you add your name? 
\item Did you disclose all resources that you have used? \\
(This includes all people, books, websites, etc. that you have consulted)
\item Did you sign that you followed the Aggie honor code? 
\item Did you solve all problems? 
\item Did you submit the pdf file of your homework?
\end{compactitem}
}



\newcommand{\problemset}[1]{\begin{center}\textbf{Problem Set
      #1}\end{center}}
\newcommand{\duedate}[1]{\begin{quote}\textbf{Due date:} Electronic
    submission of the pdf file of this homework is due on
    \textbf{#1} on canvas. \end{quote} }

\newcommand{\N}{\mathbf{N}}
\newcommand{\R}{\mathbf{R}}
\newcommand{\Z}{\mathbf{Z}}


\begin{document}
\problemset{6}
\duedate{3/10/2023 before 11:59pm}
\name{ Huy Quang Lai }
\begin{resources} Cormen, Thomas H., Leiserson, Charles E., Rivest, Ronald L., Stein, Clifford. \textit{Introduction to Algorithms}. The MIT Press.
\end{resources}
\honor

\newpage
\textbf{Make sure that you describe all solutions in your own
words, and that the work presented is your own!} Typesetting in
\LaTeX{} is required. Read the chapter on amortized analysis in our textbook. 

\begin{problem}[20 points] In the lecture, we discussed a stack with
  PUSH, POP, and MULTIPOP operations.  If the set of stack operations
  included a MULTIPUSH operation, which pushes $k$ items onto the
  stack, would the $O(1)$ bound on the amortized cost of stack
  operations continue to hold?
\end{problem}
\begin{solution}
The $O(1)$ bound on the amortized cost of stack operations would \textbf{NOT} continue to hold.

\noindent
The time complexity of such a series of operations depends on the number of pushes could be made. Since one \verb+MULTIPUSH+ needs $O(k)$ time, performing $n$ \verb+MULTIPUSH+ operations would take $O(kn)$ time, leading to amortized cost of $O(k)$.
\end{solution}

\begin{problem}[20 points]
  In the lecture, we discussed a $k$-bit counter with an INCREMENT
  operation.  Show that if a DECREMENT operation were included in the $k$-bit
  counter example, $n$ operations could cost as much as $\Theta(nk)$
  time.
\end{problem}
\begin{solution}
In the worst case \verb+INCREMENT+ needs to flip all ones to zeros. Example $0111111\to1000000$.\\
Likewise for \verb+DECREMENT+ needs to flip all zeros to ones. Example $1000000\to0111111$.
Both of these are both $\Theta(k)$ for $k$-bit counters.\\
Therefore, alternating between \verb+INCREMENT+ and \verb+DECREMENT+ is $\Theta(nk)$

\end{solution}

\begin{problem}[20 points] \label{base}
  Suppose we perform a sequence of $n$ operations on a data structure
  in which the $i$-th operation costs $i$ if $i$ is an exact power of
  $2$, and $1$ otherwise. Use aggregate analysis to determine the
  amortized cost per operation.
\end{problem}
\begin{solution}
In a sequence of $n$ operations there are $\lfloor\log_2n\rfloor+1$ exact powers of $2$. 
They are: $2^0,2^1,2^2,\cdots,2^{\lfloor\log_2n\rfloor}$\\
The total cost of the cost is a geometric sum:
\[\sum_{i=0}^{\lfloor\log_2n\rfloor}2^i=2^{\lfloor\log_2n\rfloor+1}-1\leq2^{\log_2 n+1}=2n\]
The rest of the operations are $O(1)$. Of there operations there are at most $n$.
The total cost of all operations is $T(n)\leq2n+n=3n\in O(n)$.\\
This leads to an amortized cost per operation of $O(1)$.
\end{solution}

\clearpage
\begin{problem}[20 points]
 Redo Problem~\ref{base} using an accounting method of analysis.
\end{problem}
\begin{solution}
The actual cost of the $i$th operations is
\[c_i=\begin{cases}
i & \text{if }i=2^k,k\in\mathbb{Z^+} \\
1 & \text{otherwise}
\end{cases}\]
Assigning amortized costs as follows:
\[\hat{c_i}=\begin{cases}
2 & \text{if }i=2^k,k\in\mathbb{Z^+} \\
3 & \text{otherwise}
\end{cases}\]

\noindent
Under this definition, when $i$ is an exact power of 2, the operation uses all previously accumulated credit plus one unit of its own amortized cost to pay its true cost. It then assigns the remaining unit as credit.\\

\noindent
When $i$ is not an exact power of 2, then the operation uses one unit to pay its actual cost and assigns the remaining two units as credit. This covers the actual cost of the operation.\\

\noindent
Therefore, the amortized cost of each operation is $O(1)$.    

\end{solution}

\begin{problem}[20 points]
Redo Problem~\ref{base} using a potential method of analysis.
\end{problem}
\begin{solution}
The potential function is defined as follows:
\[\Phi(D_0)=0 \text{ and } \Phi(D_i)=2i-2^{\lfloor\log_2 i\rfloor+1}+1 \text{ for } i>0\]
Since this potential function is always nonnegative, $\Phi(D_i)\geq\Phi(D_0)\forall i$

\noindent
When $i$ is an exact power of 2, the potential difference is
\[\Phi(D_i)-\Phi(D_{i-1})=(2i-2i+1)-(2(i-1)-(i-1))=2-i\]
Therefore, the amortized cost of the $i$th operation is 
\[\hat{c_i}=c_i+\Phi(D_i)-\Phi(D_{i-1})=i+2-i=2\]
\end{solution}

\noindent
When $i$ is not an exact power of 2, the potential difference is
\[\Phi(D_i)-\Phi(D_{i-1})=(2i-i+1)-(2(i-1)-(i-1))=2\]
Therefore, the amortized cost of the $i$th operation is
\[\hat{c_i}=c_i+\Phi(D_i)-\Phi(D_{i-1})=1+2=3\]

\noindent
With this result, the amortized cost of each operation is $O(1)$.

\end{document}
